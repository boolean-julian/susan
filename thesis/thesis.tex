%!TEX option = --shell-escape
\documentclass[a4paper, 11pt]{report}
\usepackage[hidelinks]{hyperref}
\usepackage{amssymb}
\usepackage{amsmath}
\usepackage{amsthm}
\usepackage[utf8]{inputenc}
\usepackage[T1]{fontenc}
\usepackage{lmodern}
\usepackage{lipsum}
\usepackage[onehalfspacing]{setspace}
\usepackage[top=35mm, left=30mm, right=30mm]{geometry}
\usepackage{graphicx}
\usepackage{svg}
\usepackage{nameref}
\usepackage{graphicx}
\usepackage{xcolor}
\usepackage{colortbl}
\usepackage{subcaption}
\usepackage{cleveref}
\usepackage{tikz}
\usepackage[main=ngerman]{babel}
\hypersetup{hypertexnames=false}

\usetikzlibrary{matrix}

\DeclareMathOperator{\rel}{\sim_R}
\renewcommand{\emph}[1]{\textit{#1}}
\newcommand{\mytitle}{\LARGE}
\newcommand{\titlespace}{\vspace{6em}}

\theoremstyle{definition}
\newtheorem{definition}{Definition}[section]
\newtheorem{example}[definition]{Beispiel}
\newtheorem{theorem}[definition]{Satz}
\newtheorem{corollary}[definition]{Korollar}
\newtheorem*{remark}{Bemerkung}
\newenvironment{myAbstract}{\section*{Abstract}}{}

\begin{document}

\newgeometry{top=35mm, left=25mm, right=25mm, bottom=35mm}
\begin{titlepage}
	\begin{center}
		\begin{minipage}{.49\textwidth}
			\flushleft
			\includesvg[width=\textwidth]{assets/uni-logo}
		\end{minipage}
		\begin{minipage}{.49\textwidth}
			\flushright
			XX XXX XX\\
			YYY YYY Y
		\end{minipage}
		\begin{minipage}{.5\textwidth}
			\begin{center}
				\vspace{2cm}
				\mytitle 		{Bachelorarbeit}\\
				\normalsize 	im Studiengang \glqq Angewandte Informatik\grqq\\
				\titlespace
				\mytitle 		{SUSAN}\\
				\normalsize 	Ein Ansatz zur Strukturerkennung in Bildern\\
				
				\titlespace		Julian Lüken\\
								\texttt{julian.lueken@stud.uni-goettingen.de}\\
				\titlespace		Institut für Numerische und Angewandte Mathematik\\
				\titlespace		Bachelor und Masterarbeiten des Zentrums für angewandte Informatik an der Georg-August-Universität Göttingen
				
				\titlespace		\today
			\end{center}
		\end{minipage}
	\end{center}
\end{titlepage}

\restoregeometry
\newgeometry{left=45mm, right=45mm}
\pagestyle{empty}

%\begin{myAbstract}
%	\lipsum[1]
%\end{myAbstract}

\tableofcontents
\pagebreak
\restoregeometry
\pagestyle{headings}

\chapter{Einführung}

\chapter{Mathematische Grundlagen}

\chapter{Der SUSAN Kantendetektor}
	Sei $I$ ein Eingangsbild. Um jedes Pixel im Bild wird eine Maske gelegt. Für unseren Zweck betrachten wir lediglich die Masken	
	\begin{center}
		$M_{37} =$
		\begin{tikzpicture}[fill=orange]
			\matrix(m)[matrix of nodes, nodes={draw, minimum size = 0.5cm}, nodes in empty cells, column sep=-\pgflinewidth,row sep=-\pgflinewidth]{
						&			&			&			&			&			&			\\
						&			&			&			&			&			&			\\
						&			&|[fill]|	&|[fill]|	&|[fill]|	&			&			\\
						&			&|[fill]|	&|[fill=yellow]|	&|[fill]|	&			&			\\
						&			&|[fill]|	&|[fill]|	&|[fill]|	&			&			\\
						&			&			&			&			&			&			\\
						&			&			&			&			&			&			\\
			};
		\end{tikzpicture}\qquad $M_{9} =$
		\begin{tikzpicture}[fill=orange]
			\matrix(m)[matrix of nodes, nodes={draw, minimum size = 0.5cm}, nodes in empty cells, column sep=-\pgflinewidth,row sep=-\pgflinewidth]{
						&			&|[fill]|	&|[fill]|	&|[fill]|	&			&			\\
						&|[fill]|	&|[fill]|	&|[fill]|	&|[fill]|	&|[fill]|	&			\\
			|[fill]|	&|[fill]|	&|[fill]|	&|[fill]|	&|[fill]|	&|[fill]|	&|[fill]|	\\
			|[fill]|	&|[fill]|	&|[fill]|	&|[fill=yellow]|	&|[fill]|	&|[fill]|	&|[fill]|	\\
			|[fill]|	&|[fill]|	&|[fill]|	&|[fill]|	&|[fill]|	&|[fill]|	&|[fill]|	\\
						&|[fill]|	&|[fill]|	&|[fill]|	&|[fill]|	&|[fill]|	&			\\
						&			&|[fill]|	&|[fill]|	&|[fill]|	&			&			\\
			};
		\end{tikzpicture}
	\end{center}
	wobei das gelbe Pixel der Mittelpunkt der Maske ist, die orangenen Pixel in der Maske und die weißen Pixel außerhalb der Maske liegen. 
	Das SUSAN-Prinzip funktioniert wie folgt:
	Für jedes Pixel in $I$, berechne die Antwort $$A(r_0) = \text{max}\{0, g - n(r_0)\}$$, wobei

	$$ n(r_0) = \sum_r c(r, r_0) $$


	$$
		c_t(r, r_0) =
			\begin{cases}
				1 	& \text{falls } |I(r) - I(r_0)| \leq t 	\\
				0 	& \text{sonst}
			\end{cases}
	$$

	$$
		c_t(r, r_0) =
			\text{exp}\bigg(-\Big(\frac{I(r) - I(r_0)}{t}\Big)^6\bigg)
	$$


	$$ A(r_0) = \text{max}\{0, g - n(r_0)\} $$

	$$ r_{\text{COG}} = \frac	{\sum_r r\,c(r,r_0)}	{\sum_r c(r,r_0)} $$



\end{document}
