%!TEX option = --shell-escape
\documentclass[a4paper, 11pt]{report}
\usepackage[hidelinks]{hyperref}
\usepackage{amssymb}
\usepackage{amsmath}
\usepackage{amsthm}
\usepackage[utf8]{inputenc}
\usepackage[T1]{fontenc}
\usepackage{lmodern}
\usepackage{lipsum}
\usepackage[onehalfspacing]{setspace}
\usepackage[top=35mm, left=30mm, right=30mm]{geometry}
\usepackage{graphicx}
\usepackage{svg}
\usepackage{nameref}
\usepackage{graphicx}
\usepackage{subcaption}
\usepackage{cleveref}
\usepackage[main=ngerman]{babel}
\hypersetup{hypertexnames=false}

\DeclareMathOperator{\rel}{\sim_R}
\renewcommand{\emph}[1]{\textit{#1}}
\newcommand{\mytitle}{\LARGE}
\newcommand{\titlespace}{\vspace{6em}}

\theoremstyle{definition}
\newtheorem{definition}{Definition}[section]
\newtheorem{example}[definition]{Beispiel}
\newtheorem{theorem}[definition]{Satz}
\newtheorem{corollary}[definition]{Korollar}
\newtheorem*{remark}{Bemerkung}
\newenvironment{myAbstract}{\section*{Abstract}}{}

\begin{document}

\newgeometry{top=35mm, left=25mm, right=25mm, bottom=35mm}
\begin{titlepage}
	\begin{center}
		\begin{minipage}{.49\textwidth}
			\flushleft
			\includesvg[width=\textwidth]{assets/uni-logo}
		\end{minipage}
		\begin{minipage}{.49\textwidth}
			\flushright
			XX XXX XX\\
			YYY YYY Y
		\end{minipage}
		\begin{minipage}{.5\textwidth}
			\begin{center}
				\vspace{2cm}
				\mytitle 		{Bachelorarbeit}\\
				\normalsize 	im Studiengang \glqq Angewandte Informatik\grqq\\
				\titlespace
				\mytitle 		{SUSAN}\\
				\normalsize 	Ein Ansatz zur Strukturerkennung in Bildern\\
				
				\titlespace		Julian Lüken\\
								\texttt{julian.lueken@stud.uni-goettingen.de}\\
				\titlespace		Institut für Numerische und Angewandte Mathematik\\
				\titlespace		Bachelor und Masterarbeiten des Zentrums für angewandte Informatik an der Georg-August-Universität Göttingen
				
				\titlespace		\today
			\end{center}
		\end{minipage}
	\end{center}
\end{titlepage}

\restoregeometry
\newgeometry{left=45mm, right=45mm}
\pagestyle{empty}

%\begin{myAbstract}
%	\lipsum[1]
%\end{myAbstract}

\tableofcontents
\pagebreak
\restoregeometry
\pagestyle{headings}

\chapter{Einführung}

\chapter{Mathematische Grundlagen}
\chapter{Der SUSAN Kantendetektor}


\end{document}
